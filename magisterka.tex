%Przykładowy plik ułatwiający złożenie projektu dyplomowego inżynierskiego.
%UWAGA: Generowany napis na stronie tytułowej o treści PROJEKT DYPLOMOWY INżYNIERSKI został zaproponowany przeze mnie i nie jest, póki co, potwierdzony przez władze wydziału. Przed ostatecznym oddaniem tak złożonej pracy należy upewnić się jaka powinna być treść tego napisu. W momencie gdy uzyskam informację na temat treści tego napisu, dokonam niezbędnych zmian w źródłach.

\documentclass[en, noamssymb]{mgr}
%opcje klasy dokumentu mgr.cls zostały opisane w dołączonej instrukcji

%poniżej deklaracje użycia pakietów, usunąć to co jest niepotrzebne
\usepackage[utf8]{inputenc} %kodowanie znaków, zależne od systemu
\usepackage[T1]{fontenc} %poprawne składanie polskich czcionek
\usepackage{lmodern}

\let\babellll\lll
\let\lll\relax

%pakiety do grafiki
\usepackage{graphicx}
\usepackage{subfigure}
\usepackage{psfrag}

%pakiety wspomagające i poprawiające składanie tabel
\usepackage{supertabular}
\usepackage{array}
\usepackage{tabularx}
\usepackage{hhline}

\usepackage{float}
\usepackage{inputenc}
\usepackage{enumitem}
\usepackage[colorlinks = true,
            linkcolor = blue,
            urlcolor  = blue,
            citecolor = blue,
            anchorcolor = blue]{hyperref}
\hypersetup{colorlinks=true, linkcolor=black}

\usepackage{amsthm}
\usepackage{indentfirst}
\newtheorem{example}{Przykład}

%pakiet wypisujący na marginesie etykiety równań i rysunków zdefiniowanych przez \label{}, chcąc wygenerować finalną wersję dokumentu wystarczy usunąć poniższą linię
%\usepackage{showlabels}

\newcommand{\wstawAng}[1]{(ang.~\emph{#1})}

%dane do złożenia strony tytułowej
\title{System wsparcia przetwarzania danych osobowych
w firmie wobec nowych wymagań wynikających z rozporządzenia RODO}
\engtitle{Personal Data Processing Support System for the Company
in the Face of New RODO Regulation Requirements}
\author{Paweł Rymer}
\supervisor{dr. inż. Jacek Mazurkiewicz}
%\guardian{dr hab. inż. Imię Nazwisko Prof. PWr, I-6} %nie używać jeśli opiekun jest tą samą osobą co prowadzący pracę

\date{2018} %standardowo u dołu strony tytułowej umieszczany jest bieżący rok, to polecenie pozwala wstawić dowolny rok

%poniżej jest lista kierunków i specjalności na wydziale elektroniki, należy wybrać właściwe lub dopisać jeśli nie ma odpowiednich
\field{Computer science (INF)}
\specialisation{Internet Engineering (INE)}

%tutaj zaczyna się właściwa treść dokumentu
\begin{document}
\bibliographystyle{abbrv} %tylko gdy używamy BibTeXa, ustawia styl bibliografii

\maketitle %polecenie generujące stronę tytułową
%\dedication{6cm}{To jest przykładowa treść opcjonalnej dedykacji, należy ją zmienić lub usunąć w całości polecenie \texttt{$\backslash$dedication}}

\tableofcontents %spis treści

\chapter{Purpose and scope of work} \label{sec:sekcjaWprowadzenie}

%Krotko i rzeczowo o tym co w pracy, dotyczy modulu ktory bedzie zaimplementowany, na koniec krotki przewodnik po rozdzialach, bez punktow.

%Używanie skrótu polskiego, wyjasnic w jednym - dwóch zdaniach.

This work presents issues related to personal data processing in the face of General Data Protection Regulation (GDPR/RODO). Upcoming changes in regulations oblige any entity that processes data to meet certain requirements. This entity is related to, among others, enterprises and companies. The more data is being processed in such entity, the more complex structure is required to manage this data. For medium and large enterprises, amount of data being processed requires the use of advanced IT systems. In the face of GDPR, such IT system should also support meeting new standard of personal data protection.\\
\indent In this work will be described the value which personal data represents,  origins of personal data protection, legal state in Poland before GDPR, what is GDPR, what it stands for and scope of changes in regulations. The available solutions will be analyzed and there will be also described proposed prototype of GDPR supporting module for existing GRC system.

\section{Description of the problem}
%Bardziej szczegolowy opis nadchodzacych zmian

On the 25th of May 2018, GDPR will take effect. Introduced changes can be divided in two ways, these more revolutionary, and these less revolutionary. These less revolutionary are basis legal concepts or rules of personal data processing which didn't actually change since current state. These more revolutionary are connected with rules to practical application \cite{giodo}. These rules assumes increasing self-reliance, but also responsibility of data administrators.\\
\indent New regulation determines way of approaching to data processing called \textit{risk based approach}. It assumes that first thing that we do during gathering and using personal data is to analyze risk that could be caused for people which data concern. Another thing is \textit{accountability rule}. It assumes that any data administrator has a duty to introduce appropriate technical and organizational mesures appling compliance with regulation requrements, but at the same time it does not describe neither any best practices nor minimal technical standards. \\

Przykladowa tresc

\chapter{Ochrona danych osobowych} \label{sec:sekcjaDaneOsobowe}
\section{Dane osobowe jako wartosc}
\section{Geneza ochrony danych osobowych}
\section{Historyczne akty prawa miedzynarodowego}
\section{Status w Polsce przed wejsciem w zycie RODO}

\chapter{RODO} \label{sec:sekcjaRODO}
\section{Nowe podejscie do ochrony danych osobowych}
\section{Zakres przetwarzanych informacji}
\section{Nowe obowiazki informacyjne}
\section{Uprawnienia osob, ktorych dane dotycza}
\section{Zgoda na przetwarzanie danych osobowych}
\section{Zabezpieczenia}
\section{Dokumentacja przetwarzania danych}
\section{Privacy by design i privacy by default}
\section{Ocena skutkow dla ochrony danych}
\section{Dane osobowe dzieci}
\section{Automatyczne przetwarzanie danych oparte na profilowaniu}
\section{Naruszenia ochrony danych}
\section{Inspektor danych osobowych}
\section{Transgraniczne przetwarzanie danych}
\section{Powierzenie danych}
\section{Podnoszenie wiedzy na temat ogolnego rozporzadzenia}

\chapter{Analiza komercyjnych rozwiazan z zakresu przetwarzania danych osobowych} \label{sec:sekcjaAnalizaRozwiazan}
\section{RSA Archer}
\section{Microsoft GDPR}
\section{SAP}

\chapter{Prototyp modulu smartGDPR wspierajacy zgodnosc z RODO} \label{sec:sekcjaOpisPrototypu}
\section{Rejestr danych przetwarzania}
\section{...}

\chapter{Wnioski} \label{sec:sekcjaWnioski}

\chapter{Podsumowanie}
\label{sec:Podsumowanie}

\addcontentsline{toc}{chapter}{Bibliography} %utworzenie w spisie treści pozycji Bibliografia
\bibliography{bibliografia} % wstawia bibliografię korzystając z pliku bibliografia.bib - dotyczy BibTeXa, jeżeli nie korzystamy z BibTeXa należy użyć otoczenia thebibliography

\end{document}